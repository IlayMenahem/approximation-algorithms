\section*{Problem 4: Generalized Steiner Forest (GSF)}
Let $G = (V, E)$ be a connected undirected graph and $T_1, T_2, \dots, T_t \subseteq V$. Let $w : E \to \R_{\ge 0}$ be a non-negative weight function over the edges. A generalized Steiner forest is a subset $E' \subseteq E$, such that for every $u, v \in T_i$ ($1 \le i \le t$) there is a path in $E'$ connecting $u$ and $v$. In the generalized Steiner forest problem the objective is to find a generalized Steiner forest of minimum weight.
(a) Let $T = \bigcup_{i=1}^t T_i$. Assume $|T_i| \ge 2$ for every $1 \le i \le t$ and consider $w' : E \to \R$ defined as follows. For an edge $(u, v)$, $w'(u, v) = |\{u, v\} \cap T|$. Show that any minimal generalized Steiner forest of $G$ is a 2-approximation with respect to the weight function $w'$. Note: a generalized Steiner forest $E'$ of $G$ is minimal if there is no $F \subsetneq E'$ which is also a generalized Steiner forest of $G$.
(b) Suggest a 2-approximation algorithm for the generalized Steiner forest problem using the local ratio technique.

\subsection*{(a) Minimal GSF is a 2-approximation for $w'$}
\textbf{Claim:} Any minimal generalized Steiner forest $E'$ of $G$ is a 2-approximation with respect to the weight function $w'$.

\begin{lemma}
    For any $T$ and $E$ the following equality holds: \[ w'(E) = \sum_{v \in T} \deg_{E}(v) \]
    
    \begin{proof}
        \[ w'(E) = \sum_{(u,v) \in E} w'(u,v) = \sum_{(u,v) \in E} |\{u,v\} \cap T| = \sum_{(u,v) \in E} \left( \mathbf{1}_{u \in T} + \mathbf{1}_{v \in T} \right) \] \[ = \sum_{v \in T} |\{(u,v) \in E\}| = \sum_{v \in T} \deg_{E}(v) \]
    \end{proof}
\end{lemma}


\begin{proof}
    \textbf{Lower Bound on $OPT$:} Consider an optimal solution $E^*$. Since $|T_i|\ge 2$ for every $i$, for each terminal $v\in T$ there exists a group $T_i\ni v$ and another terminal $u\in T_i\setminus\{v\}$. Feasibility implies a path between $u$ and $v$ in $E^*$, hence $\deg_{E^*}(v)\ge 1$. Therefore,
    \[ OPT = w'(E^*) = \sum_{v \in T} \deg_{E^*}(v) \ge \sum_{v \in T} 1 = |T| \]

    \textbf{Upper Bound on $E'$:} Let $E'$ be a minimal generalized Steiner forest of $G$, and consider the forest $G'=(V,E')$.
    Since $E'$ is minimal, no connected component of $G'$ that contains terminals can contain a cycle (otherwise we could remove an edge on the cycle and preserve all required terminal-to-terminal connections inside that component). Hence, every such component is a tree.

    Fix a connected component $H$ of $G'$ that contains at least one terminal. Let $T_H := T \cap V(H)$ and let $S_H := V(H)\setminus T_H$ be the non-terminals of $H$.
    We claim that no vertex of $S_H$ can be a leaf in the tree $H$: if some $x\in S_H$ had degree 1 in $H$, then removing its unique incident edge would not disconnect any pair $u,v\in T_i$ (since $x$ is not a terminal), contradicting minimality.
    Therefore, every $x\in S_H$ satisfies $\deg_H(x)\ge 2$, so
    \[
        \sum_{x\in S_H}\deg_H(x)\ \ge\ 2|S_H|.
    \]
    Since $H$ is a tree, \(\sum_{v\in V(H)}\deg_H(v)=2(|V(H)|-1)=2(|T_H|+|S_H|-1)\). Thus,
    \[
        \sum_{v\in T_H}\deg_H(v)
        = 2(|T_H|+|S_H|-1) - \sum_{x\in S_H}\deg_H(x)
        \le 2(|T_H|+|S_H|-1) - 2|S_H|
        = 2|T_H|-2.
    \]
    Summing over all terminal-containing components (say there are $c\ge 1$ of them) gives
    \[
        w'(E')=\sum_{v\in T}\deg_{E'}(v)\ \le\ \sum_{H}(2|T_H|-2)\ =\ 2|T|-2c\ \le\ 2|T|.
    \]

    \textbf{Ratio:} Combining the bounds, we have:
    \[ w'(E') \le 2|T| \le 2 OPT \]
    Thus, $E'$ is a 2-approximation with respect to the weight function $w'$.
\end{proof}

\subsection*{(b) Local Ratio Algorithm}
\subsubsection*{Algorithm}
We use a component-based approach. Let $C(v)$ denote the connected component containing $v$ in the current forest $F$. A component $C$ is called \textbf{active} if there exists a group $T_i$ such that $C \cap T_i \neq \emptyset$ and $T_i \not\subseteq C$ (i.e., the component separates a terminal group).

There are two primitive procedures that are used in the algorithm:

\begin{itemize}
    \item $\textsc{isGsf}(F, \{T_1, \dots, T_k\})$: This procedure checks if the forest $F$ is a generalized Steiner forest.
    \begin{algorithm}[H]
    \caption{isGsf($F, \{T_1, \dots, T_k\}$)}
    \begin{algorithmic}[1]
        \State $G_F \gets (V,F)$
        \For{$i \in \{1, \dots, k\}$}
            \If{not \textsc{isConnected}($G_F, T_i$)}
                \State \Return false
            \EndIf
        \EndFor
        \State \Return true
    \end{algorithmic}
    \end{algorithm}
    The complexity of this algorithm is $O(k \cdot (|V| + |F|))$.

    \item \textsc{PruneForest}$(F, S, \{T_1, T_2, \dots, T_t\})$: This procedure makes the forest $F$ minimal by removing any edge whose deletion does not break the generalized Steiner property.
    \begin{algorithm}[H]
    \caption{PruneForest($F, S, \{T_1, T_2, \dots, T_t\}$)}
    \begin{algorithmic}[1]
        \While{$S \neq \emptyset$}
            \State $e \leftarrow \textsc{Pop}(S)$
            \If{\textsc{isGsf}($F \setminus \{e\}$, $\{T_1, \dots, T_t\}$)}
                \State $F \leftarrow F \setminus \{e\}$
            \EndIf
        \EndWhile
    \end{algorithmic}
    \end{algorithm}
    The complexity of this algorithm is $O(|S| \cdot k (|V| + |F|))$ in the worst case.
\end{itemize}

\begin{algorithm}[H]
\caption{2-Approximation for Generalized Steiner Forest}
\begin{algorithmic}[1]
    \Function{GSF}{$G = (V, E), \{T_1, \dots, T_t\}, w$}
        \State $F \leftarrow \emptyset$
        \State $S \leftarrow Stack(\emptyset)$
        \While{NOT \textsc{isGsf}($F, \{T_1, \dots, T_t\}$)}
            \State Let $\mathcal{C}$ be the set of active components in $(V, F)$
            \State Define $w'(u,v) = |\{x \in \{u,v\} : C(x) \in \mathcal{C} \text{ and } C(u) \neq C(v)\}|$
            \State $\epsilon \gets \min \{ w(e)/w'(e) \mid e \in E, w'(e) > 0 \}$
            \State $w(e) \leftarrow w(e) - \epsilon \cdot w'(e)$ for all $e \in E$
            \State Add all edges with $w(e) = 0$ to $F$ (if connecting distinct components)
            \State Push added edges onto $S$
        \EndWhile
        \State \textsc{PruneForest}$(F, S, \{T_1, \dots, T_t\})$
        \State \Return $F$
    \EndFunction
\end{algorithmic}
\end{algorithm}

\subsubsection*{Proof of Correctness}
We claim that assuming the graph is connected, the algorithm terminates and returns a valid Generalized Steiner Forest $F$.

\begin{lemma}
    For any input for which the graph is connected, the loop runs at most $|E|$ times.
\end{lemma}

\begin{proof}
    In each iteration, we select $\epsilon$ based on the active edges. This ensures at least one edge reaches weight 0. Edges with zero weight are added to $F$ (if not already present) and never removed during the loop. Since $|E|$ is finite, the loop runs at most $|E|$ times.
\end{proof}

\begin{enumerate}
    \item \textbf{Termination:} By the Lemma above, the \textbf{while} loop runs at most $|E|$ times. Since each iteration takes finite time, the loop must terminate. Furthermore, since the graph $G$ is connected, a valid GSF exists. The loop continues adding edges until the condition $\textsc{isGsf}(F, \{T_i\})$ is met.
    \item \textbf{Feasibility:} The \textbf{while} loop terminates only when $\textsc{isGsf}(F, \{T_i\})$ returns true, and the pruning phase maintains this property.
\end{enumerate}

\subsubsection*{Approximation Analysis}
\begin{theorem}
    Let $F$ be the forest returned by the algorithm and let $OPT$ be an optimal generalized Steiner forest for the original weights $w$. Then $w(F) \le 2 \cdot w(OPT)$.
\end{theorem}

\begin{proof}
    Let $F$ denote the output \emph{after} the pruning step; then $F$ is feasible and minimal.
    The weights are decomposed into iterations: at iteration $j$ we subtract $\epsilon_j w'_j$ from the current weights (where $w'_j$ is defined by the active components in that iteration), so overall
    \[
        w \;=\; w_{\mathrm{final}} \;+\; \sum_j \epsilon_j w'_j.
    \]
    Since the algorithm adds only edges of (current) weight $0$, the returned forest satisfies $w_{\mathrm{final}}(F)=0$. Therefore,
    \[
        w(F)=\sum_j \epsilon_j\, w'_j(F)
        \qquad\text{and}\qquad
        w(OPT)\ge \sum_j \epsilon_j\, w'_j(OPT).
    \]
    It thus suffices to show $w'_j(F) \le 2 \cdot w'_j(OPT)$ for each iteration $j$.

    Consider iteration $j$ with active components $\mathcal{C}_j$. We view the graph as a contracted multigraph where each component of $F$ is a node. In this view:
    \begin{itemize}
        \item The "terminals" are the nodes corresponding to $\mathcal{C}_j$.
        \item The weight function $w'_j(e)$ counts how many endpoints of $e$ are "terminals" (Active Components).
    \end{itemize}

    \textbf{Lower Bound:} Any feasible solution $OPT$ must connect every active component $C \in \mathcal{C}_j$ to some other vertex (otherwise the terminals in $T_i$ separated by $C$ remain disconnected). Thus, in the contracted view, every node in $\mathcal{C}_j$ has degree $\ge 1$ in $OPT$. Summing degrees:
    \[ w'_j(OPT) = \sum_{C \in \mathcal{C}_j} \deg_{OPT}(C) \ge |\mathcal{C}_j| \]

    \textbf{Upper Bound:} The final forest $F$ is minimal. Consider the contracted view where each component of $(V, F_j)$ becomes a single node. Edges internal to $F_j$-components disappear under contraction, and minimality implies the contracted graph is a forest.

    Any leaf in this contracted forest must be an active component:
    \begin{itemize}
        \item If a leaf $C$ were inactive, then for every group $T_i$ either $C\cap T_i=\emptyset$ or $T_i\subseteq C$. Hence no connectivity requirement forces $C$ to connect to other components. Removing the unique edge incident to $C$ would keep feasibility, contradicting minimality.
    \end{itemize}
    Since all leaves are active components (i.e., in $\mathcal{C}_j$), applying the degree-sum argument from Part (a):
    \[ w'_j(F) \le 2 |\mathcal{C}_j| - 2 \]

    \textbf{Conclusion:} $w'_j(F) \le 2 |\mathcal{C}_j| \le 2 \cdot w'_j(OPT)$. Summing over all iterations proves the theorem.
\end{proof}

\subsubsection*{Complexity Analysis}
The algorithm runs in polynomial time.
\begin{enumerate}
    \item \textbf{Number of Iterations:} As proven in the Lemma, the \textbf{while} loop executes at most $|E|$ times.
    
    \item \textbf{Cost per Iteration:} Inside the loop, the most expensive operations are checking \textsc{isGsf} and updating edge weights. \textsc{isGsf} takes $O(t(|V|+|F|))$ time. Updating weights and finding $\epsilon$ takes $O(|E|)$.
    
    \item \textbf{Pruning Phase:} The pruning step iterates through the stack $S$, which contains at most $|E|$ edges. Thus the complexity of the pruning step is $O(|S| \cdot t (|V| + |F|)) = O(|E| \cdot t (|V| + |E|))$.
\end{enumerate}

Since the number of iterations is bounded by $|E|$ and the work per iteration is polynomial, the total running time is polynomial. Specifically, a loose upper bound is $O(|E| \cdot t \cdot (|V|+|E|))$, which is polynomial in the input size.
