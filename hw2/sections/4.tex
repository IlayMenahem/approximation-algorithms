\section*{Problem 4: Generalized Steiner Forest (GSF)}
Let $G = (V, E)$ be a connected undirected graph and $T_1, T_2, \dots, T_t \subseteq V$. Let $w : E \to \R_{\ge 0}$ be a non-negative weight function over the edges. A generalized Steiner forest is a subset $E' \subseteq E$, such that for every $u, v \in T_i$ ($1 \le i \le t$) there is a path in $E'$ connecting $u$ and $v$. In the generalized Steiner forest problem the objective is to find a generalized Steiner forest of minimum weight.
(a) Let $T = \bigcup_{i=1}^t T_i$. Assume $|T_i| \ge 2$ for every $1 \le i \le t$ and consider $w' : E \to \R$ defined as follows. For an edge $(u, v)$, $w'(u, v) = |\{u, v\} \cap T|$. Show that any minimal generalized Steiner forest of $G$ is a 2-approximation with respect to the weight function $w'$. Note: a generalized Steiner forest $E'$ of $G$ is minimal if there is no $F \subsetneq E'$ which is also a generalized Steiner forest of $G$.
(b) Suggest a 2-approximation algorithm for the generalized Steiner forest problem using the local ratio technique.

\subsection*{(a) Minimal GSF is a 2-approximation for $w'$}
\textbf{Claim:} Any minimal generalized Steiner forest $E'$ of $G$ is a 2-approximation with respect to the weight function $w'$.

\begin{lemma}
    For any $T$ and $E$ the following equality holds: \[ w'(E) = \sum_{v \in T} \deg_{E}(v) \]
    
    \begin{proof}
        \[ w'(E) = \sum_{(u,v) \in E} w'(u,v) = \sum_{(u,v) \in E} |\{u,v\} \cap T| = \sum_{(u,v) \in E} \left( \mathbf{1}_{u \in T} + \mathbf{1}_{v \in T} \right) \] \[ = \sum_{v \in T} |\{(u,v) \in E\}| = \sum_{v \in T} \deg_{E}(v) \]
    \end{proof}
\end{lemma}


\begin{proof}
    \textbf{Lower Bound on $OPT$:} Consider a optimal solution $E^*$. Since $\forall 1 \le i \le t, |T_i| \ge 2$, we know that any $v \in T$ must be connected to at least one other vertex in $T_i$ via some edge in $E^*$. Thus, each vertex in $T$ contributes at least 1 to the total weight of $E^*$. Therefore, we have:
    \[ OPT = w'(E^*) = \sum_{v \in T} \deg_{E^*}(v) \ge \sum_{v \in T} 1 = |T| \]

    \textbf{Upper Bound on $E'$:} Let $E'$ be a minimal generalized Steiner forest of $G$, and consider the forest $G'=(V,E')$.
    Since $E'$ is minimal, no connected component of $G'$ that contains terminals can contain a cycle (otherwise we could remove an edge on the cycle and preserve all required terminal-to-terminal connections inside that component). Hence, every such component is a tree.

    Fix a connected component $H$ of $G'$ that contains at least one terminal. Let $T_H := T \cap V(H)$ and let $S_H := V(H)\setminus T_H$ be the non-terminals of $H$.
    We claim that no vertex of $S_H$ can be a leaf in the tree $H$: if some $x\in S_H$ had degree 1 in $H$, then removing its unique incident edge would not disconnect any pair $u,v\in T_i$ (since $x$ is not a terminal), contradicting minimality.
    Therefore, every $x\in S_H$ satisfies $\deg_H(x)\ge 2$, so
    \[
        \sum_{x\in S_H}\deg_H(x)\ \ge\ 2|S_H|.
    \]
    Since $H$ is a tree, \(\sum_{v\in V(H)}\deg_H(v)=2(|V(H)|-1)=2(|T_H|+|S_H|-1)\). Thus,
    \[
        \sum_{v\in T_H}\deg_H(v)
        = 2(|T_H|+|S_H|-1) - \sum_{x\in S_H}\deg_H(x)
        \le 2(|T_H|+|S_H|-1) - 2|S_H|
        = 2|T_H|-2.
    \]
    Summing over all terminal-containing components (say there are $c\ge 1$ of them) gives
    \[
        w'(E')=\sum_{v\in T}\deg_{E'}(v)\ \le\ \sum_{H}(2|T_H|-2)\ =\ 2|T|-2c\ \le\ 2|T|.
    \]

    \textbf{Ratio:} Combining the bounds, we have:
    \[ w'(E') \le 2|T| \le 2 OPT \]
    Thus, $E'$ is a 2-approximation with respect to the weight function $w'$.
\end{proof}

\subsection*{(b) Local Ratio Algorithm}
\subsubsection*{Algorithm}
here are three primitive procedures that are used in the algorithm:

\begin{itemize}
    \item $\textsc{isGsf}(F, \{T_1, \dots, T_k\})$: This procedure checks if the forest $F$ is a generalized Steiner forest.
    \begin{algorithm}[H]
    \caption{isGsf($F, \{T_1, \dots, T_k\}$)}
    \begin{algorithmic}[1]
        \State $G_F \gets (V,F)$
        \For{$i \in \{1, \dots, k\}$}
            \If{not \textsc{isConnected}($G_F, T_i$)}
                \State \Return false
            \EndIf
        \EndFor
        \State \Return true
    \end{algorithmic}
    \end{algorithm}
    The complexity of this algorithm is $O(k \cdot (|V| + |F|))$.
    
    \item $\textsc{UpdateActiveSet}(e=(u,v),T)$: This procedure propagates the "active" status to new nodes (the "infection" step).
    \begin{itemize}
        \item If $u \in T$ and $v \notin T$, update $T \leftarrow T \cup \{v\}$.
        \item If $v \in T$ and $u \notin T$, update $T \leftarrow T \cup \{u\}$.
        \item If both or neither are in $T$, do nothing.
    \end{itemize}
    The complexity of this algorithm is $O(\log(|T|))$.

    \item \textsc{PruneForest}$(F, S, \{T_1, T_2, \dots, T_t\})$: This procedure makes the forest $F$ minimal by removing any edge whose deletion does not break the generalized Steiner property.
    \begin{algorithm}[H]
    \caption{PruneForest($F, S, \{T_1, T_2, \dots, T_t\}$)}
    \begin{algorithmic}[1]
        \While{$S \neq \emptyset$}
            \State $e \leftarrow \textsc{Pop}(S)$
            \If{\textsc{isGsf}($F \setminus \{e\}$, $\{T_1, \dots, T_t\}$)}
                \State $F \leftarrow F \setminus \{e\}$
            \EndIf
        \EndWhile
    \end{algorithmic}
    \end{algorithm}
    The complexity of this algorithm is $O(|S| \cdot k (|V| + |F|))$ in the worst case.
\end{itemize}

\begin{algorithm}[H]
\caption{2-Approximation for Generalized Steiner Forest using Local Ratio}
\begin{algorithmic}[1]
    \Function{GSF}{$G = (V, E), \{T_1, T_2, \dots, T_t\}, w$}
        \State $F \leftarrow \emptyset$
        \State $\mathcal{T} \leftarrow \bigcup_{i=1}^k T_i$
        \State $S \leftarrow Stack(\emptyset)$
        \While{NOT \textsc{isGsf}($F, \{T_i\}$)} 
            \State $E_{active} \leftarrow \{ (u,v) \in E \mid \{u,v\} \cap \mathcal{T} \neq \emptyset \}$
            \State $w'(e) \gets \begin{cases} |\{u,v\} \cap \mathcal{T}| & \text{if } e = (u,v) \in E_{active} \\ 0 & \text{otherwise} \end{cases}$
            \State $\epsilon \gets \min_{e \in E_{active}} \frac{w(e)}{w'(e)}$ 
            \State $w(e) \leftarrow w(e) - \epsilon \cdot w'(e)$ 
            \State $E^* \gets \{e \in E_{active} \mid w(e) = 0\} \setminus F$
            \State $F \leftarrow F \cup E^*$ 
            \State Push each $e \in E^*$ onto $S$ 
            \State for each $e \in E^*$ do: \textsc{UpdateActiveSet}($e, \mathcal{T}$)
        \EndWhile
        \State \textsc{PruneForest}$(F, S, \{T_1, T_2, \dots, T_t\})$
        \State \Return $F$
    \EndFunction
    \end{algorithmic}
\end{algorithm}

\subsubsection*{Proof of Correctness}
We claim that the algorithm terminates and returns a valid Generalized Steiner Forest $F$.

before we prove the correctness of the algorithm, we should prove the following
\begin{lemma}
    For any input for which the graph is connected, the loop runs at most $|E|$ times.
\end{lemma}

\begin{proof}
    In each iteration, we select $\epsilon = \min_{e \in E_{active}} \frac{w(e)}{w'(e)}$. This ensures at least one edge $e^*$ in $E_{active}$ reaches weight 0. Edges with zero weight are added to $F$ (if not already present) and never removed during the loop. Since $|E|$ is finite and each iteration effectively processes at least one new edge towards inclusion in $F$, the loop runs at most $|E|$ times.
\end{proof}

\begin{enumerate}
    \item \textbf{Termination:} By the Lemma above, the \textbf{while} loop runs at most $|E|$ times. Since each iteration takes finite time, the loop must terminate. Furthermore, since the graph $G$ is connected, a valid GSF exists (e.g., the spanning tree). The loop continues adding edges until the condition $\textsc{isGsf}(F, \{T_i\})$ is met. In the worst case, $F$ grows to include all edges in $E$, which is definitely a GSF. Thus, the algorithm terminates with a valid solution.

    \item \textbf{Feasibility (Phase 1):} The \textbf{while} loop terminates only when $\textsc{isGsf}(F, \{T_i\})$ returns true. Therefore, the set $F$ obtained after the loop satisfies the Generalized Steiner Forest property.

    \item \textbf{Feasibility (Phase 2 - Pruning):} The pruning phase iterates through the edges in $S$ and removes an edge $e$ from $F$ only if $\textsc{isGsf}(F \setminus \{e\}, \{T_i\})$ remains true. This explicitly maintains the invariant that $F$ is a valid Generalized Steiner Forest. Thus, the final set $F$ returned by the algorithm is a valid solution.
\end{enumerate}

\subsubsection*{Approximation Analysis}

\begin{theorem}
    Let $F$ be the forest returned by the algorithm and let $OPT$ be an optimal generalized Steiner forest for the original weights $w$. Then
    \[
        w(F)\ \le\ 2\cdot w(OPT).
    \]
\end{theorem}

\begin{proof}
We prove the claim via the local ratio technique.

\textbf{1. Weight decomposition.}
Let $w^{(0)}:=w$ be the initial weight function, and let the algorithm run for $k$ iterations.
In iteration $j$ it chooses $\epsilon_j$ and defines a nonnegative auxiliary weight function $w'_j$ (supported on the currently active edges), and updates
\[
    w^{(j)}(e)\ :=\ w^{(j-1)}(e) - \epsilon_j\cdot w'_j(e).
\]
By the choice $\epsilon_j=\min_{e\in E_{active}} \frac{w^{(j-1)}(e)}{w'_j(e)}$ we maintain $w^{(j)}(e)\ge 0$ for all edges $e$.
Unrolling the updates yields, for every edge $e$,
\[
    w(e)\ =\ \sum_{j=1}^k \epsilon_j w'_j(e)\ +\ w^{(k)}(e),
    \qquad\text{with } w^{(k)}(e)\ge 0.
\]

\textbf{2. The solution cost has no residual part.}
Edges are added only when they become tight, i.e., when their current weight drops to 0.
Weights only decrease, so any edge that is ever added has final residual weight $w^{(k)}(e)=0$.
The pruning phase only removes edges, hence for the final returned forest $F$ we have
\[
    w(F)\ =\ \sum_{e\in F} w(e)\ =\ \sum_{e\in F}\sum_{j=1}^k \epsilon_j w'_j(e)\ =\ \sum_{j=1}^k \epsilon_j\, w'_j(F).
\]

\textbf{3. Lower bound on $OPT$ under the decomposition.}
For any feasible forest $X$ (in particular $X=OPT$),
\[
    w(X)\ =\ \sum_{j=1}^k \epsilon_j w'_j(X) + w^{(k)}(X)\ \ge\ \sum_{j=1}^k \epsilon_j w'_j(X).
\]
So \(w(OPT)\ge \sum_{j=1}^k \epsilon_j w'_j(OPT)\).

\textbf{4. Bounding each component via part (a).}
Fix an iteration $j$. The algorithm’s \emph{active set} $\mathcal{T}_j$ is obtained by repeatedly “infecting” vertices along edges whose current weight became 0 and were added to $F$.
Therefore every vertex in $\mathcal{T}_j$ lies in the same zero-weight connected component (at iteration $j$) as some original terminal.
Contract all zero-weight edges added up to iteration $j$; this yields a contracted instance in which the images of the original terminal sets still have size at least 2 whenever they impose a nontrivial constraint.
In that contracted instance, the function $w'_j$ is exactly of the form from part (a): \(w'_j(e)=|\{u,v\}\cap T^{(j)}|\) for the union \(T^{(j)}\) of the contracted terminals.

By part (a), \emph{any minimal feasible generalized Steiner forest} (for those contracted terminal sets) is a 2-approximation with respect to $w'_j$.
The pruning step makes $F$ minimal, hence we can apply part (a) to conclude:
\[
    w'_j(F)\ \le\ 2\cdot w'_j(OPT).
\]

\textbf{5. Summing up.}
Multiplying by $\epsilon_j$ and summing over all iterations,
\[
    w(F)\ =\ \sum_{j=1}^k \epsilon_j w'_j(F)
    \ \le\ \sum_{j=1}^k 2\epsilon_j w'_j(OPT)
    \ \le\ 2\, w(OPT).
\]
\end{proof}

\subsubsection*{Complexity Analysis}
The algorithm runs in polynomial time.
\begin{enumerate}
    \item \textbf{Number of Iterations:} As proven in the Lemma, the \textbf{while} loop executes at most $|E|$ times.
    
    \item \textbf{Cost per Iteration:} Inside the loop, the most expensive operations are checking \textsc{isGsf} and updating edge weights. \textsc{isGsf} takes $O(k(|V|+|E|))$ time. Updating weights and finding $\epsilon$ takes $O(|E|)$.
    
    \item \textbf{Pruning Phase:} The pruning step iterates through the stack $S$, which contains at most $|E|$ edges. thus the complexity of the pruning step is $O(|S| \cdot k (|V| + |F|)) = O(|E| \cdot k (|V| + |E|))$.
\end{enumerate}

Since the number of iterations is bounded by $|E|$ and the work per iteration is polynomial, the total running time is polynomial. Specifically, a loose upper bound is $O(|E| \cdot k \cdot (|V|+|E|))$, which is polynomial in the input size.
